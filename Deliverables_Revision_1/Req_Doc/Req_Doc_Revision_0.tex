%%%%%%%%%%%%%%%%%%%%%%%%%%%%%%%%%%%%%%%%%
% University Assignment Title Page 
% LaTeX Template
% Version 1.0 (27/12/12)
%
% This template has been downloaded from:
% http://www.LaTeXTemplates.com
%
% Original author:
% WikiBooks (http://en.wikibooks.org/wiki/LaTeX/Title_Creation)
%
% License:
% CC BY-NC-SA 3.0 (http://creativecommons.org/licenses/by-nc-sa/3.0/)
% 
% Instructions for using this template:
% This title page is capable of being compiled as is. This is not useful for 
% including it in another document. To do this, you have two options: 
%
% 1) Copy/paste everything between \begin{document} and \end{document} 
% starting at \begin{titlepage} and paste this into another LaTeX file where you 
% want your title page.
% OR
% 2) Remove everything outside the \begin{titlepage} and \end{titlepage} and 
% move this file to the same directory as the LaTeX file you wish to add it to. 
% Then add \input{./title_page_1.tex} to your LaTeX file where you want your
% title page.
%
%%%%%%%%%%%%%%%%%%%%%%%%%%%%%%%%%%%%%%%%%

%----------------------------------------------------------------------------------------
%	PACKAGES AND OTHER DOCUMENT CONFIGURATIONS
%----------------------------------------------------------------------------------------

\documentclass[12pt]{article}
\usepackage[hidelinks]{hyperref}
\hypersetup{linktoc=all}
\begin{document}

\begin{titlepage}

\newcommand{\HRule}{\rule{\linewidth}{0.5mm}} % Defines a new command for the horizontal lines, change thickness here

\center % Center everything on the page
 
%----------------------------------------------------------------------------------------
%	HEADING SECTIONS
%----------------------------------------------------------------------------------------

\textsc{\LARGE McMaster University}\\[1.5cm] % Name of your university/college
\textsc{\Large Software Project Management}\\[0.5cm] % Major heading such as course name
\textsc{\large SFWR ENG 3XA3}\\[0.5cm] % Minor heading such as course title

%----------------------------------------------------------------------------------------
%	TITLE SECTION
%----------------------------------------------------------------------------------------

\HRule \\[0.4cm]
{ \huge \bfseries Requirements Document}\\[0.4cm] % Title of your document
\HRule \\[1.5cm]
 
%----------------------------------------------------------------------------------------
%	AUTHOR SECTION
%----------------------------------------------------------------------------------------



% If you don't want a supervisor, uncomment the two lines below and remove the section above
\Large \emph{Author:}\\
Mohammad \textsc{Naveed} \textbf{1332196} \\ % Your name
Josh \textsc{Voskamp} \textbf{1319352} \\
Stephan \textsc{Arulthasan} \textbf{1308004} \\[3cm]
%----------------------------------------------------------------------------------------
%	DATE SECTION
%----------------------------------------------------------------------------------------

{\large \today}\\[3cm] % Date, change the \today to a set date if you want to be precise

%----------------------------------------------------------------------------------------
%	LOGO SECTION
%----------------------------------------------------------------------------------------

%\includegraphics{Logo}\\[1cm] % Include a department/university logo - this will require the graphicx package
 
%----------------------------------------------------------------------------------------

\vfill % Fill the rest of the page with whitespace

\end{titlepage}

\newpage
\tableofcontents
\newpage
\listoftables
\addcontentsline{toc}{section}{List of Tables}
\newpage
\listoffigures
\addcontentsline{toc}{section}{List of Figures}
\newpage

\section*{Revision History}
\addcontentsline{toc}{section}{Revision History}
\begin{table}[!htbp]
	\centering
	\begin{tabular}{ | p{2cm} | p{2cm}| p{6cm} | p{4cm}|}
		\hline
		Rev. No. & Rev. Date & Description & Author \\\hline
		0 & Oct 5 2015 & Created Document & Mohammad Naveed \\\hline
		0 & Oct 5 2015 & Added Off the Shelf Software & Josh Voskamp \\\hline
		0 & Oct 7 2015 & Added Functional Requirements & Stephan Arulthasan\\\hline
		0 & Oct 7 2015 & Improve Formatting & Josh Voskamp \\\hline
		0 & Oct 9 2015 & Improved Functional Requirements & Stephan Arulthasan \\\hline
		0 & Oct 9 2015 & Added Non-Functional Requirements & Mohammad Naveed \\\hline
		0 & Oct 9 2015 & Create Project Issues & Josh Voskamp \\\hline
		0 & Oct 9 2015 & Add to Project Issues & Mohammad Naveed \\\hline
		0 & Oct 9 2015 & Complete Project Issues & Stephan Arulthasan \\\hline
		0 & Oct 9 2015 & Finalize Requirements Documentation & Josh Voskamp \\\hline
		
	\end{tabular}
	\caption{Revision History}
\end{table}
\newpage

\section{Purpose of the Project}
\subsection{What is the problem being solved?} 
\par\indent\indent It is commonly known that technology, although serving countless purposes, is a large source of distraction to many people. More specifically, online applications, although highly entertaining and addicting, 
are very limited in cognitive stimulation. This is highly problematic as we are enabling a culture of absent minded technological engagement.
\subsection{Why is this an Important Problem?} 
\par\indent\indent According to Jane McGonigal, a well known and world renowned game designer; we spend 3 
billion hours a week playing video games. That is a lot of time that many people argue could be spent better, and 
that is what 2048 aims to accomplish. More and more people are playing video games everyday and 2048 is a 
fun and challenging game that tests the users' mathematical as well as their spatial intelligence. This allows 2048 
to be fun, yet still be brain enhancing. Since the target audience for this game is so large, we can take advantage 
of this by providing users an option to spend their gaming time in a way thats beneficial mentally while still being 
entertained.
\subsection{Context of the problem} 
\par\indent\indent Everyone experiences idle time in their day; this could be waiting for an appointment, a class, a 
bus or for friends. This game is intended to appeal to everyone looking for a more entertaining way to spend their 
idle time. The complexity of the game is meant to provide a challenge so that the user does not feel like they are 
wasting time, but using their time constructively. The game will be playable on all of the three major operating 
systems, OSX, Windows, and Linux with possible future expansion to mobile devices.

\section{Stakeholders}
\subsection{Client}
\begin{itemize}
\item N/A
\end{itemize}
\subsection{Customer}
\begin{itemize}
\item Gamer
\end{itemize}
\subsection{Other Stakeholders}
\begin{itemize}
\item Developers and Testers
\end{itemize}

\section{Users of the Product}
\subsection{Hands-On Users of the Product}
\par\indent\indent Any person with any computer running the Java Runtime Environment can use our game to 
relieve stress and build on their mathematical and spatial intelligence. 
\subsection{Priorities assigned to Users}
\textbf{Primary Users:} Windows, Mac OSX, and LINUX users
\\
\textbf{Secondary Users:} Developers, testers and supervisors

\subsection{User Participation}
\begin{itemize}
\item User just has to play the game to participate
\end{itemize}
\subsection{Maintainence Users and Testers}
\begin{itemize}
\item Developers/Testers
\end{itemize}

\section{Project Constraints}
\subsection{Mandated Constraints}
\textbf{Description:} The product shall operate using Windows, Mac OSX and 
LINUX\\
\textbf{Rationale:} Users do not want to change operating systems just for the 
game and it will be more convenient for them.\\
\textbf{Description:} Product should be easy to use for people over the age 
of 10\\
\textbf{Rationale:} The product does require a certain level of mathematical 
and spatial intelligence which is why users under 10 years of age may find it 
difficult to use. 
\subsection{Off the Shelf Software}
\par\indent\indent The open source project that is being improved can be found
at \\ https://github.com/bulenkov/2048.
This implementation of 2048 is based off the original 2048 game created by 
Gabriele Cirulli which was also based off a small clone of 1024. 
https://github.com/gabrielecirulli/2048.
\subsection{Anticipated Workplace Enviroment}
This application is expected to be used at home, in the workplace, at school, and in the public.\newline
\subsection{Schedule Constraints}
Requirements Document Revision 0	\hfill	October 9 \newline
Proof of Concept Demonstration \hfill		October 26-28 \newline
Final Demonstration \hfill				November 30 - December 4 \newline

\section{Functional Requirements}
\subsection{Scope of the Project}
Everyone experiences idle time in their day; this could be waiting for an 
appointment, a class, a bus or for friends. That is a lot of time that many 
people argue could be spent better, and that is what 2048 aims to accomplish. 
2048 is a game in which the user uses the arrow keys to combine alike tiles in 
an attempt to reach the "2048" tile.\\

\subsection{Functional and Data Requirements}
\textbf {Requirement \#:1} \indent\textbf {Requirement Type: 9}\\\\
\textbf {Description:} The user must be able to start a game.\\
\textbf {Rational:} In order for the player to play the game, it has to initialize first.\\
\textbf {Fit Criterion:} User successively starts game so they can play. \\\\

\textbf {Requirement \#:2} \indent\textbf {Requirement Type: 9} \\\\
\textbf {Description:} The user must be able to restart the game.\\
\textbf {Rational:}The user may want to restart if they made a bad move or have already won or lost the game. \\
\textbf {Fit Criterion:}Users successfully restarts the game. \\\\

\textbf {Requirement \#:3} \indent\textbf {Requirement Type: 9}\\\\
\textbf {Description:} The user must be able to exit the game.\\
\textbf {Rational:} The user may want to exit the session at any time.\\
\textbf {Fit Criterion:} User successfully exits game. \\\\

\textbf {Requirement \#:4} \indent\textbf {Requirement Type: 9}\\\\
\textbf {Description:}The user must be able to make valid moves. \\
\textbf {Rational:} In order to play and win the game, the user must be able to move the tiles around. \\
\textbf {Fit Criterion:}User successfully makes a valid move within the gameboard, where the tiles join correctly and score counter increases by the value created. \\\\

\textbf {Requirement \#:5} \indent\textbf {Requirement Type: 9} \\\\
\textbf {Description:} The user must be able to win the game.\\
\textbf {Rational:} The player is working towards making the 2048 tile. \\
\textbf {Fit Criterion:} User successfully makes the 2048 tile and the game ends. \\\\

\textbf {Requirement \#:6} \indent\textbf {Requirement Type: 9}\\\\
\textbf {Description:}The user must be able to lose the game. \\
\textbf {Rational:} It is possible to make a bad sequence of moves in the game so that all the tiles are filled and 
no more moves can be created.\\
\textbf {Fit Criterion:} The user fills up all the tile spaces and the game ends.\\\\

\textbf {Requirement \#:7} \indent\textbf {Requirement Type: 9} \\\\
\textbf {Description:} The product must display the game score.\\
\textbf {Rational:} The user will be able to rank themselves against other players and try to beat their own high 
score.\\
\textbf {Fit Criterion:} Product successfully displays score based on moves made by user.\\\\

\textbf {Requirement \#:8} \indent\textbf {Requirement Type: 9} \\\\
\textbf {Description:} The product must notify the user if they win or 
lose.\\
\textbf {Rational:} The user must be able to tell if they successfully win the game. Also, the tiles on the screen 
may be filled up but that doesn't necessarily mean they lost because there can be moves available. \\
\textbf {Fit Criterion:} The game displays a win notification when the user reaches the 2048 tile and a lose 
notification when the user fills up all the tile spaces with no moves left. \\\\

\section{Non-Functional Requirements}

\subsection{Look \& Feel Requirements}
\begin{itemize}
\item Appearance Requirements
\begin{itemize} 
\item Board size must be 4x4
\item The interface must be intuitive
\end{itemize}
\end{itemize}
\subsection{Usability Requirements}
\begin{itemize}
\item Ease of use Requirements
\begin{itemize}
\item Product shall be easy to use for anyone who is older than the age of 10
\end{itemize}
\item Personalization and Internationalization Requirements
\begin{itemize}
\item N/A
\end{itemize}
\item Ease of Learning of Requirements
\begin{itemize}
\item The product shall be easy to learn for anyone who is older than the age of 10
\end{itemize}
\end{itemize}
\subsection{Performance Requirements}
\begin{itemize}
\item Speed requirements
\begin{itemize}
\item The game must start in under 2 seconds
\item Any interaction with the user and the product during gameplay must have a maximum response time of 2 seconds
\end{itemize}

\item{Precision Requirements}
\begin{itemize}
\item All values on the tiles must be integers 
\item The value of the smallest tile must be 2
\item The value of the largest tile must be 2048
\item The value on all tiles must be a multiple of 2
\end{itemize}

\item{Reliability \& Availability Requirements}
\begin{itemize}
\item The product shall be available for use 24 hours a day for 365 days of the year
\end{itemize}
\item{Safety Critical Requirements}
\begin{itemize}
\item N/A
\end{itemize}
\item{Capacity Requirements}
\begin{itemize}
\item The product shall cater to 1 user on the machine
\end{itemize}
\end{itemize}


\subsection{Operational Requirements}
\begin{itemize}
\item Expected Physical Environment
\begin{itemize}
\item The product is to be used by gamers at home, in the workplace, at school, and in the public 
\end{itemize}
\item Expected Technological Environment
\begin{itemize}
\item The product shall be available on LINUX, Mac OSX, and Windows operating systems
\item The environment must have the Java Runtime Environment (JRE)
\end{itemize}
\item Partner Applications
\begin{itemize}
\item N/A
\end{itemize}
\end{itemize}

\subsection{Maintainability and Support Requirements}
\begin{itemize}
\item Maintenance Requirements 
\begin{itemize}
\item Make it easy to fix any potential bugs in the code
\end{itemize}

\item Supportability Requirements
\begin{itemize}
\item The product will not be supported, however the source code will be available for examination or enhancement.
\end{itemize}
\item Adaptability Requirements 
\begin{itemize}
\item The product is expected to run on LINUX, Mac OSX, and Windows operating systems
\end{itemize}
\end{itemize}

\subsection{Security Requirements}
\begin{itemize}
\item N/A
\end{itemize}

\subsection{Cultural Requirements}
\begin{itemize}
\item The product shall not use icons that could be considered offensive in any of our market countries.
\end{itemize}

\subsection{Compliance Requirements}
\begin{itemize}
\item N/A
\end{itemize}

\section{Project Issues}
\subsection{Open Issues}
\begin{itemize}
	\item Creating a GUI compatible with various screen resolutions
	\item Animation for the sliding of the tiles
	\item Add a FAQ option
\end{itemize}

\subsection{Off-the-Shelf Solutions}
\begin{itemize}
	\item https://github.com/bulenkov/2048 The Open-source Java project being 
	improved.
	\item https://github.com/gabrielecirulli/204 The Original Open-source 
	project for the game 2048, It was originally implemented in JavaScript.
\end{itemize}

\subsection{New Problems}
\subsubsection{Effects on the Current Environment}
N/A
\subsubsection{Effects on the Installed Systems}
N/A
\subsubsection{Potential User Problems}
N/A
\subsubsection{Limitations in the Anticipated Implementation Environment That 
May Inhibit the New Product}
N/A
\subsubsection{Follow-Up Problems}
N/A
\subsection{Costs}
There are no direct monetary costs associated with this project, but about half a year of development
time will be required.\\

\subsection{User Documentation and Training}
User's will provided with information on program use via a FAQ option which, 
when selected, will open a dialog box detailing general functionality of the 
program. Beyond the help document, user?s familiar with casual computer use 
should require no further training.\\

\subsection{Tasks}
\begin{itemize}
	\item Revise requirements document.
	\item Create a test plan.
	\item Demonstrate a proof of concept.
	\item Draw up design documents.
	\item Revision 0 project demonstration.
	\item Create a user guide for the project.
	\item Write up a test report
	\item Final revision project demonstration.
	\item Write final revisions to documentation.
\end{itemize}

\subsection{Waiting Room}
There are no plans to introduce new releases of this product. If that were to change, new features would include 
an animated GUI and an online high score list. 

\subsection{Ideas for Solutions}
\begin{itemize}
	\item Use Java Swing and AWT for GUI development
\end{itemize}

\end{document}