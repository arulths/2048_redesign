%%%%%%%%%%%%%%%%%%%%%%%%%%%%%%%%%%%%%%%%%
% University Assignment Title Page
% LaTeX Template
% Version 1.0 (27/12/12)
%
% This template has been downloaded from:
% http:\\www.LaTeXTemplates.com
%
% Original author:
% WikiBooks (http:\\en.wikibooks.org/wiki/LaTeX/Title_Creation)
%
% License:
% CC BY-NC-SA 3.0 (http:\\creativecommons.org/licenses/by-nc-sa/3.0/)
% %%%%%%%%%%%%%%%%%%%%%%%%%%%%%%%%%%%%%%%%
\documentclass[12pt]{article}
\usepackage{xcolor}

\begin{document}
	\sloppy
	
	\begin{titlepage}
		\color{red}
		\newcommand{\HRule}{\rule{\linewidth}{0.5mm}} % Defines a new command 
		%for the
		%horizontal lines, change thickness heReve
		
		\center % Center everything on the page
		
		%----------------------------------------------------------------------------------------
		%	HEADING SECTIONS
		%----------------------------------------------------------------------------------------
		
		\textsc{\LARGE McMaster University}\\[1.5cm] % Name of your 
		%university/college
		\textsc{\Large Software Project Management}\\[0.5cm] % Major heading 
		%such as course name
		\textsc{\large SFWR ENG 3XA3}\\[0.5cm] % Minor heading such as course 
		%title
		
		%----------------------------------------------------------------------------------------
		%	TITLE SECTION
		%----------------------------------------------------------------------------------------
		
		\HRule \\[0.4cm]
		{ \huge \bfseries Problem Statement}\\[0.4cm] % Title of your 
		%document
		\HRule \\[1.5cm]
		
		%----------------------------------------------------------------------------------------
		%	AUTHOR SECTION
		%----------------------------------------------------------------------------------------
		
		
		
		% If you don't want a supervisor, uncomment the two lines below and 
		%remove the section above
		\Large \emph{Authors:}\\
		Mohammad \textsc{Naveed} \textbf{1332196} \\ % Your name
		Josh \textsc{Voskamp} \textbf{1319352} \\
		Stephan \textsc{Arulthasan} \textbf{1308004} \\[3cm]
		%----------------------------------------------------------------------------------------
		%	DATE SECTION
		%----------------------------------------------------------------------------------------
		
		{\large \today}\\[3cm] % Date, change the \today to a set date if you 
		%want to be precise
		
		%----------------------------------------------------------------------------------------
		%	LOGO SECTION
		%----------------------------------------------------------------------------------------
		
		%\includegraphics{Logo}\\[1cm] % Include a department/university logo - 
		%%this will require the graphicx package
		
		%----------------------------------------------------------------------------------------
		
		\vfill % Fill the rest of the page with whitespace
		
\end{titlepage}

%What problem are you trying to solve?
It is commonly known that technology, although serving countless purposes, is a large source of distraction to many people. More specifically, online applications, although highly entertaining and addicting, are very limited in cognitive stimulation. This is highly problematic as we are enabling a culture of absent minded technological engagement. \par \vspace{5mm}

%Why is this an important problem?
According to Jane McGonigal, a well known and world renowned game designer; we spend 3 billion hours a week playing video games. That is a lot of time that many people argue could be spent better, and that is what 2048 aims to accomplish. More and more people are playing video games everyday and 2048 is a fun and challenging game that tests the users' mathematical as well as their spatial intelligence. This allows 2048 to be fun, yet still be brain enhancing. Since the target audience for this game is so large, we can take advantage of this by providing users an option to spend their gaming time in a way thats beneficial mentally while still being entertained. \par \vspace{5mm}

%What is the context of the problem you are solving?
Everyone experiences idle time in their day; this could be waiting for an appointment, a class, a bus or for friends. This game is intended to appeal to everyone looking for a more entertaining way to spend their idle time. The complexity of the game is meant to provide a challenge so that the user does not feel like they are wasting time, but using their time constructively. 
The game will be playable on all of the three major operating systems, OSX, Windows, and Linux with possible future expansion to mobile devices. \vspace{10mm}

\textbf{Revision 0} September 28 2015: First draft of the problem statement. 

\end{document}