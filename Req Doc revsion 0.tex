
\documentclass[12pt]{amsart}
\usepackage{geometry} % see geometry.pdf on how to lay out the page. There's lots.
\geometry{a4paper} % or letter or a5paper or ... etc
% \geometry{landscape} % rotated page geometry

% See the ``Article customise'' template for come common customisations

\title{Software Requirements Documentation}
\author{Mohammad Naveed, Josh Voskamp, Stephan Arulthasan}


%%% BEGIN DOCUMENT
\begin{document}

\maketitle
\newpage
\tableofcontents
\newpage

\section{Purpose of the Project}
\subsection{What is the problem being solved?} 
.\par
It is commonly known that technology, although serving countless purposes, is a large source of distraction to many people. More specifically, online applications, although highly entertaining and addicting, are very limited in cognitive stimulation. This is highly problematic as we are enabling a culture of absent minded technological engagement
\subsection{Why is this an Important Problem?} 
.\par
According to Jane McGonigal, a well known and world renowned game designer; we spend 3 billion hours a week playing video games. That is a lot of time that many people argue could be spent better, and that is what 2048 aims to accomplish. More and more people are playing video games everyday and 2048 is a fun and challenging game that tests the users' mathematical as well as their spatial intelligence. This allows 2048 to be fun, yet still be brain enhancing. Since the target audience for this game is so large, we can take advantage of this by providing users an option to spend their gaming time in a way thats beneficial mentally while still being entertained.
\subsection{Context of the problem} 
.\par Everyone experiences idle time in their day; this could be waiting for an appointment, a class, a bus or for friends. This game is intended to appeal to everyone looking for a more entertaining way to spend their idle time. The complexity of the game is meant to provide a challenge so that the user does not feel like they are wasting time, but using their time constructively. The game will be playable on all of the three major operating systems, OSX, Windows, and Linux with possible future expansion to mobile devices.

\section{Stakeholders}
\subsection{Client}
.\par
\begin{itemize}
\item N/A
\end{itemize}
\subsection{Customer}
.\par
\begin{itemize}
\item Gamer
\end{itemize}
\subsection{Other Stakeholders}
.\par
\begin{itemize}
\item Developers and Testers
\end{itemize}

\section{Users of the Product}
\subsection{Hands-On Users of the Product}
.\par
Any person with any computer running the Java Runtime Environment can use our game to relieve stress and build on their mathematical and spatial intelligence. 
\subsection{Priorities assigned to Users}
.\par
\textbf{Primary Users:} Windows, Mac OSX, and LINUX users
.\par
\textbf{Secondary Users:} Developers, testers and supervisors
.\par
\subsection{User Participation}
\begin{itemize}
\item User just has to play the game to participate
\end{itemize}
\subsection{Maintainence Users and Testers}
\begin{itemize}
\item Developers/Testers
\end{itemize}

\section{Project Constraints}
\subsection{Mandated Constraints}
.\newline
\textbf{Descripton:} The product shall operate using Windows, Mac OSX and LINUX
\textbf{Rationale:} Users do not want to change operating systems just for the game and it will be more convenient for them. 
.\newline
.\newline
\textbf{Descripton:} The product should be easy to use for people over the age of 10
\textbf{Rationale:} The product does require a certain level of mathematical and spatial intelligence which is why users under 10 years of age may find it difficult to use. 

\subsection{Off the Shelf Software}
.\newline
The open source project that is being improved can be found at https://github.com/bulenkov/2048.
This implementation of 2048 is based off the original 2048 game created by Gabriele Cirulli which was also based off a small clone of 1024. https://github.com/gabrielecirulli/2048.


\end{document}