
\documentclass[12pt]{article}
\usepackage{geometry} % see geometry.pdf on how to lay out the page. There's lots.
\geometry{a4paper} % or letter or a5paper or ... etc
% \geometry{landscape} % rotated page geometry

% See the ``Article customise'' template for come common customisations

\title{Software Requirements Documentation}
\author{Mohammad Naveed, Josh Voskamp, Stephan Arulthasan}


%%% BEGIN DOCUMENT
\begin{document}

\maketitle
\newpage
\tableofcontents
\newpage

\section{Purpose of the Project}
\subsection{What is the problem being solved?} 
\par\indent\indent It is commonly known that technology, although serving 
countless purposes, is a large source of distraction to many people. More 
specifically, online applications, although highly entertaining and addicting, 
are very limited in cognitive stimulation. This is highly problematic as we are 
enabling a culture of absent minded technological engagement.
\subsection{Why is this an Important Problem?} 
\par\indent\indent According to Jane McGonigal, a well known and world renowned game designer; we spend 3 billion hours a week playing video games. That is a lot of time that many people argue could be spent better, and that is what 2048 aims to accomplish. More and more people are playing video games everyday and 2048 is a fun and challenging game that tests the users' mathematical as well as their spatial intelligence. This allows 2048 to be fun, yet still be brain enhancing. Since the target audience for this game is so large, we can take advantage of this by providing users an option to spend their gaming time in a way thats beneficial mentally while still being entertained.
\subsection{Context of the problem} 
\par\indent\indent Everyone experiences idle time in their day; this could be waiting for an appointment, a class, a bus or for friends. This game is intended to appeal to everyone looking for a more entertaining way to spend their idle time. The complexity of the game is meant to provide a challenge so that the user does not feel like they are wasting time, but using their time constructively. The game will be playable on all of the three major operating systems, OSX, Windows, and Linux with possible future expansion to mobile devices.

\section{Stakeholders}
\subsection{Client}
\begin{itemize}
\item N/A
\end{itemize}
\subsection{Customer}
\begin{itemize}
\item Gamer
\end{itemize}
\subsection{Other Stakeholders}
\begin{itemize}
\item Developers and Testers
\end{itemize}

\section{Users of the Product}
\subsection{Hands-On Users of the Product}
\par\indent\indent Any person with any computer running the Java Runtime Environment can use our game to relieve stress and build on their mathematical and spatial intelligence. 
\subsection{Priorities assigned to Users}
\textbf{Primary Users:} Windows, Mac OSX, and LINUX users
\\
\textbf{Secondary Users:} Developers, testers and supervisors

\subsection{User Participation}
\begin{itemize}
\item User just has to play the game to participate
\end{itemize}
\subsection{Maintainence Users and Testers}
\begin{itemize}
\item Developers/Testers
\end{itemize}

\section{Project Constraints}
\subsection{Mandated Constraints}
\textbf{Description:} The product shall operate using Windows, Mac OSX and LINUX
\textbf{Rationale:} Users do not want to change operating systems just for the game and it will be more convenient for them.\\
\textbf{Description:} The product should be easy to use for people over the age of 10
\textbf{Rationale:} The product does require a certain level of mathematical and spatial intelligence which is why users under 10 years of age may find it difficult to use. 
\subsection{Off the Shelf Software}
\par\indent\indent The open source project that is being improved can be found
at\\ https://github.com/bulenkov/2048.
This implementation of 2048 is based off the original 2048 game created by Gabriele Cirulli which was also based off a small clone of 1024. https://github.com/gabrielecirulli/2048.
\subsection{Anticipated Workplace Enviroment}
This application is expected to be used at home, in the workplace, at school, and in the public.\newline
\subsection{Schedule Constraints}
Requirements Document Revision 0	\hfill	October 9 \newline
Proof of Concept Demonstration \hfill		October 26-28 \newline
Final Demonstration \hfill				November 30 - December 4 \newline

\section{Functional Requirements}
\subsection{Scope of the Project}
Everyone experiences idle time in their day; this could be waiting for an 
appointment, a class, a bus or for friends. That is a lot of time that many 
people argue could be spent better, and that is what 2048 aims to accomplish. 
2048 is a game in which the user uses the arrow keys to combine alike tiles in 
an attempt to reach the "2048" tile.\\

\section{Project Issues}
\subsection{Open Issues}
\begin{itemize}
	\item Creating a GUI compatible with various screen resolutions
	\item Animation for the sliding of the tiles
	\item Add a FAQ option
\end{itemize}

\subsection{Off-the-Shelf Solutions}
\begin{itemize}
	\item https://github.com/bulenkov/2048 The Open-source Java project being 
	improved.
	\item https://github.com/gabrielecirulli/204 The Original Open-source 
	project for the game 2048, It was originally implemented in JavaScript.
\end{itemize}

\subsection{New Problems}
\subsubsection{Effects on the Current Environment}
Not Applicable
\subsubsection{Effects on the Installed Systems}
Not Applicable
\subsubsection{Potential User Problems}
Not Applicable
\subsubsection{Limitations in the Anticipated Implementation Environment That 
May Inhibit the New Product}
Not Applicable
\subsubsection{Follow-Up Problems}
Not Applicable

\subsection{Ideas for Solutions}
\begin{itemize}
	\item Use Java Swing and AWT for GUI development
\end{itemize}

\end{document}